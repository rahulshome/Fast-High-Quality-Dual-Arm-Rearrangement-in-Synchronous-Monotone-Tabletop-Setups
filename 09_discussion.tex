The current work demonstrates the underlying structure of synchronized \dual-arm rearrangement and proposes an \milp formulation, as well as a scalable algorithm \algo that provides fast, high quality solutions. Existing efficient solvers for reductions of the dual-arm problem made \algo effective. 

\cameraready
{
Work beyond the methods presented in this paper has explored the $k-$arm case. The matching sub-problem ceases to have effective solvers for $k>2$, and heuristics might need to be considered for feasible solutions. Effective approximate or lazy solutions to the $k-$arm case can also prove to be power heuristics to the general multi-arm task planning~\cite{shome2019multiarm} challenge. Regrasp reasoning and the use of intermediate locations can extend this work to non-monotone problem instances. The incorporation of manipulation and grasp reasoning with \revisions{real-world object geometries} can also extend such solvers to more cluttered environments. This work serves as a stepping stone in building towards these rich problem domains.
}

% It can be shown however that efficient approximate solvers for more than 2 arms would be hard to design.
% Let us assume a $ \rho $-Approx algorithm, $\mathbb{A}$ exists for the problem of rearrangement with 3 arms. Assume a hypothetical manipulator that incurs a constant transit cost on a bounded plane. The problem of an optimal set of object assignments reduce to 3D-Matching. $ \rho $-Approx $\mathbb{A}$ should return a solution to 3DM. Since 3DM is hard to approximate, $\mathbb{A}$ does not exist.\kiril{The connection with 3DM is not immediate, and I don't think that there's enough information provided here in order to make sense of it. I suggest to omit this description, especially if space is limited.}