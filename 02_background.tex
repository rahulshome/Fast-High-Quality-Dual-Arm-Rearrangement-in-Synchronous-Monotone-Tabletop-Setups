%Related efforts:
%
%\begin{itemize}
%\item multi-robot planning and coordination
%\item manipulation task planning in general
%\item rearrangement 
%\item combinatorial literature
%\end{itemize}
The current work dealing with dual-arm object rearrangement touches upon the challenging intersection of a variety of rich bodies of prior work. It is closely related to multi-robot planning and coordination. The challenge with multi-body planning is the high dimensionality of the configuration space. %Early work has suggested to slightly reduce the~\cite{aronov1999motion} tried dimensionality reduction strategies. 
Optimal strategies were developed for simpler instances of the problem \cite{solovey2015motion}, although in general the problem is known to be computationally hard~\cite{solovey2016hardness}. Decentralized approaches~\cite{van2005prioritized} also used velocity tuning~\cite{leroy1999multiple} to deal with these difficult instances. General multi-robot planning tries to plan for multiple high-dimensional platforms \cite{wagner2012probabilistic,Gharbi:2009fu} using sampling-based techniques. 
Recent advances provide scalable~\cite{SoloveySH16:ijrr} and asymptotically optimal~\cite{Dobson:2017aa} sampling based frameworks. 
%Assembly planning \cite{halperin2000general,sundaram2001disassembly} is another related problem dealing with multi-body planning.

%The discrete version of the multi-body problem, also known as the "pebble motion problem" has seen a lot of work \cite{kornhauser1984coordinating,auletta1999linear,goraly2010multi}. For such discrete problems on a pebble graph, feasibility times are linear and solution times are polynomial. Optimality is still challenging even in these setups. Heuristics have been used to provide high quality solutions \cite{wagner2012probabilistic,sharon2015conflict}. 

In some cases, by restricting the input of the problem to a certain type, it is possible to cast known hard instances of a problem as related algorithmic problems which have efficient solvers. For instance, unlabeled multi-robot motion planning can be reduced to pebble motion on graphs~\cite{abhs-unlabeled14}; pebble motion can be reduced to network flow~\cite{yu2016optimal}; and single-arm object rearrangement can be cast as a traveling salesman problem~\cite{193}. These provide the inspiration to closely inspect the structure of the problem to derive efficient solutions.


%\subsection{Combinatorial Literature}
%Our problem can be formulated in a variety of related classes of problems in combinatorial literature. Most of the approaches assume that all the edge weights are known, ie. the motion planning for all the edges has to be performed before any operation on the graph that uses the weights.

%\textbf{The Traveling Salesman Problem}:
In this work we leverage a connection between dual-arm rearrangement and two combinatorial problems: (1) optimal matching~\cite{edmonds1965maximum} and (2) TSP. On the surface the problem seems closely related to multi-agent TSP.
%This can itself be tweaked with assumptions on same or different start and goal positions, symmetric or asymmetric metrics, directed or undirected graphs or posing the problem as an optimization
A seminal paper~\cite{frederickson1976approximation} provides the formulation for k-TSP, with solutions that split a single tour.
Prior work \cite{rathinam2006matroid} has formulated the problem of multi-start to multi-goal k-TSP as an optimization task. Some work~\cite{friggstad2013multiple} deals with asymmetric edge weights which are more relevant to the problems of our interest.
 
%\textbf{Errand Scheduling}:
%If the object transfer is seen as an errand \cite{slavik1997errand} then a graph constructed with nodes being object-arm assignments and edges implying the order of coordinated execution. 
%There has been prior work \cite{garg2000polylogarithmic} that operates over the assignment graph(where pairwise assignments are nodes), an object's group consists of sets of nodes that include an object

The multi-arm rearrangement can also be posed as an instance of multi vehicle pickup and delivery (PDP)~\cite{parragh2008survey}. 
%Most of the work has been motivated by ride-sharing applications and timed task scheduling, so there are a lot of variations that deal with time windows but this line of research provides some ILP formulations for the problem.
Prior work~\cite{coltin2014multi} has applied the PDP problem to robots, taking into account time windows and robot-robot transfers.
Some seminal work~\cite{lenstra1981complexity,savelsbergh1995general} has also explored its complexity, and concedes to the hardness of the problem, while others have studied cost bounds~\cite{TrePavFra13}. ILP formulations~\cite{savelsbergh1995general} have also been proposed. 
%The existing work that delves into the combinatorial challenges of the problem, do not account for the costs incurred due to coordination of the agents during task execution. 
%Pickup and Delivery problems under track contention \cite{caricato2003parallel}, makes an attempt to reason about the coordination of the robots over solutions obtained with Tabu Search.
Typically this line of work ignores coordination costs, though some methods~\cite{caricato2003parallel} reason about it once candidate solutions are obtained.

% There has been a a lot of work in motion planning for manipulators and movable objects. 
Navigation among movable objects deals with the combinatorial challenges of multiple objects~\cite{wilfong1991motion,van2009path} and has been shown to be a hard problem, and extended to manipulation applications~\cite{stilman2007manipulation}. Despite a lot of interesting work on challenges of manipulation and grasp planning, the current work shall make assumptions that avoid complexities arising from them. Manipulators opened the applications of rearrangement planning~\cite{ben1998practical,ota2004rearrangement}, including instances where objects can be grasped only once or monotone~\cite{stilman2007manipulation}, as well as non-monotone instances~\cite{havur2014geometric,srivastava2014combined}. Efficient solutions to assembly planning problems~\cite{Wilson:1994fk,Halperin:2000uq} typically assumes monotonicity, as without it the problem becomes much more difficult. Recent work has dealt with the hard instances of task planning~\cite{berenson2011task,cohen2014single} and rearrangement planning~\cite{krontiris2015dealing,krontiris2016efficiently,193}. Sampling-based task planning has made a recent push towards guarantees of optimality~\cite{vega2016asymptotically,schmitt2017optimal}. These are broader approaches that are invariant to the combinatorial structure of the application domain. The current work draws inspiration from these varied lines of research.

\commentadd{
General task planning methods are unaware of the underlying structure studied in this work. Single-arm rearrangement solutions will also not effective in this setting. The current work tries to bridge this gap and provide insights regarding the structure of dual-arm rearrangement. Under assumptions that enable this study, an efficient solution emerges for this problem.
}