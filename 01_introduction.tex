%Advances in robotics have been pushing the boundaries of the
%application of robots to automation. There has been a lot of interest
%in robotic picking in this context\cite{correll2016analysis} and
%significant recent efforts have been made for facing the challenges of
%manipulation and grasping.

\IEEEPARstart{A}{utomation} tasks in industrial and service robotics, such as product packing or sorting, often require sets of objects to be arranged in
specific poses on a planar surface. Efficient and high-quality
single-arm solutions have been proposed for such
setups \cite{193}. The proliferation of robot arms, however, including \dual-arm setups, implies that industrial settings can utilize multiple robots in the same workspace (Fig~\ref{fig:two_arms}). This work explores a) the benefits and trade-offs of \revisions{using two arms for solving object rearrangement instead of a} single-arm, b) the combinatorial challenges involved and c) computationally efficient, high-quality and scalable methods.

%While the benefits seem obvious, even for two arms, the challenge
%lies in examining the combinatorial sub-structure of this integrated
%task and motion planning problem.
 
%The current work focuses on rearrangement problems involving two
%arms.  Compared to the sort of solutions that yield high quality
%single arm solutions to these problems, it will be demonstrated that
%significant solution quality improvements can be made, which
%motivates the need to find effective algorithms that can solve the
%dual-arm rearrangement problem.  We seek to bound the cost of $k$-arm
%manipulation.  Note that $c_t$ may be used to capture distance
%(energy) cost or time cost.

%Due to lying at the confluence of various challenging problem
%domains, namely multi-robot motion planning, object rearrangement
%task planning, and combinatorial search, the problem at hand demands
%closer inspection through these lenses. It is desirable to obtain
%some guarantees about the benefits of using more than one arm, before
%endeavoring to study the problem.

A major motivation for this work is that the coordinated use of multiple arms \revisions{to move multiple objects at the same time} can
result in significant improvements in efficiency \revisions{of object rearrangement solutions}. This arises from the following argument.

% \vspace{-.1in}
\begin{observation}\label{l:k-arm-lower-bound}
There are classes of tabletop rearrangement problems, where a $k$-arm
($k \ge 2$) solution can be arbitrarily better than the optimal
single-arm one.
% \vspace{-.3in}
\end{observation}

For instance, assume two arms that have full (overhand) access to a 
unit square planar tabletop. There are $n$ objects on the table, 
divided into two groups of $\frac{n}{2}$ each. Objects in each group 
are $\varepsilon$-close to each other and to their goals. Let the 
distance between the two groups be on the order of $1$, i.e., the two 
groups are at opposite ends of the table. The initial position of each 
end-effector is $\varepsilon$-close to a group of objects. Let the cost 
of each pick and drop action be $c_{pd}$, while moving the 
end-effector costs $c_t$ per unit distance. Then, the $2$-arm 
cost is no more than $2nc_{pd} + 2n\varepsilon c_t $.  A single arm 
solution costs at least $2nc_{pd} + (2n-1)\varepsilon c_t + 
c_t$. If $c_{pd}$ and $\varepsilon$ are sufficiently small, the $2$-arm 
cost can be arbitrarily better than the single-arm one. The argument 
also extends to $k$-arms relative to $(k-1)$-arms.



\begin{figure}[t]
% \vspace{-0.15in}
	\centering
	\includegraphics[width=0.48\textwidth]{figures/two_arms}
% 	\vspace{-0.1in}
	\caption{\Dual-arm setups that can utilize
	algorithms proposed in this work.}
	\label{fig:two_arms}
	\vspace{-0.25in}
\end{figure}



In most practical setups, the expectation is that a $2$-arm solution
will be close to half the cost (e.g., time-wise) of the single-arm 
case, which is a desirable improvement. While there 
is coordination overhead, the best $2$-arm solution cannot do worse; simply 
let one of the arms carry out the best single arm solution while the 
other remains outside the workspace. \revisions{This means whenever there is enough available space for $k$ arms, it can be said:}

% \vspace{-.1in}
\begin{observation}\label{l:2-arm-no-worse}
For any rearrangement problem, the best $k$-arm ($k \ge 2$) solution
cannot be worse than an optimal single arm solution.
% \vspace{-.1in}
\end{observation}


The above observations motivate the development of scalable algorithmic
tools for such \dual-arm rearrangement instances. This work considers
certain relaxations to achieve this objective. In particular,
monotone tabletop instances are considered, where the start and goal
object poses do not overlap.  Furthermore, the focus is on
synchronized execution of pick-and-place actions by the two arms. i.e., where the two arms simultaneously transfer (different) objects, or simultaneously move towards picking the next objects.
Theoretical arguments and experimental evaluation indicate that this does not significantly degrade solution
quality.

\revisions{It should be noted here that the solutions presented in this paper go beyond purely reasoning about task allocation and scheduling of objects to robots. The \dual-arm object rearrangement solution needs to consider geometric interactions and collisions between robots and obstacles through coordinated motion planning. Information about these real-world interactions need to be considered to yield solutions that are ultimately valid and collision-free.}



The first contribution is the study of the combinatorial structure of
synchronous, monotone \dual-arm rearrangement. Then, a mixed integer linear programming (\milp) model is
proposed that achieves optimal coordinated solutions in this domain.
The proposed efficient algorithm {\tt Tour Over Matching} (\algo) significantly improves in scalability. \algo\ first optimizes the
cost of object transfers and assigns objects to the two arms by
solving an optimal matching problem. It minimizes the cost of
moves from an object's goal to another object's start pose per
arm as a secondary objective by employing a {\tt TSP} solution. Most
of the computation time is spent on the many calls to an underlying motion planner that coordinates the two arms. A lazy
evaluation strategy is proposed, where motion plans are evaluated and validated for candidate solution sequences, as opposed to performing this expensive motion plan computation for the entire search space. This results in significant computational improvement and
reduces the calls to the motion planner.  An analysis
studies the expected improvement in solution quality
versus the single arm case, as well as the expected cost overhead from a
synchronous solution. 

Finally, experiments for i) a simple planar picker setting, and ii) two 7-DOF Kuka arms, demonstrate a nearly two-fold
improvement against the single arm case for the proposed approach in practice. The algorithm exhibits close to optimal solutions and good scalability. 
The lazy evaluation strategy significantly improves computation costs for both the optimal and proposed methods. 

\tase{
The current archival version is an extension of previous work \cite{shome2018dualarm}. In particular, several additions have been made to provide a more holistic study of the problem:
\begin{enumerate}
    % \item Additional explanation and visual support is provided for understanding the algorithmic steps, especially regarding Section \ref{sec:approximation}.
    \item A detailed illustration and expository text is provided for describing the algorithmic steps, especially regarding Section \ref{sec:approximation}.
	\item Further analysis has been performed to study the problem (Section \ref{sec:cost_bounds}). Specifically, the result describing the asymptotic cost bound for asynchronous operation has been extended to the $k$-arm case. Additionally, a more detailed proof is provided for the asymptotic cost bound estimate in the case of synchronized \dual-arm operation.
% 	\item New data (Section \ref{sec:smoothing}) is provided that study the effect of post-processing solutions so as to minimize delays introduced due to the synchronization assumption. 
	\item The solutions from the benchmarks were post-processed so as to minimize delays introduced due to the synchronization assumption, and the data from this \textit{smoothing} operation is provided in Section \ref{sec:smoothing}.
	\item A benchmark (Section \ref{sec:reachability}) has been included to demonstrate the efficacy of the proposed approach in problem domains where initial or target poses of objects do not lie in the shared workspace.
\end{enumerate}
}

%The computational costs of both the optimal
%and the proposed method are significantly improved from the lazy
%evaluation strategy. 
% When the synchronous solutions are smoothed to
% asynchronous ones, there is not significance difference in overall
% path quality. \kiril{This statement is unclear. What is the benefit from performing this step?}
