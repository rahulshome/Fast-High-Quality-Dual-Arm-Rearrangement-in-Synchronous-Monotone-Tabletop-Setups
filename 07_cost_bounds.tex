
% This section studies the dual arm costs in the randomized unit tabletop setting, where $c_t$ is the cost measure per unit distance.  For the $2$-arm setting, assume for simplicity that each arm's volume is represented as a disc of some radius $r$. For obtaining a $2$-arm solution, first partition the $n$ objects randomly into two sets of $\frac{n}{2}$ objects each. Then, obtain the two initial solutions similar to the single arm case.  It is expected (Eq.~\ref{eq:single-cost-simple}) that these two halves should add up to approximately $(c_{pd} + 0.52c_t)n$. The reasoning follows Lemma \ref{lem:transferdomination}, which states that transfers dominate the cost of the solution.

% \begin{theorem}
% 	\textit{A $2$-arm solution can have an asymptotic improvement of $\frac{1}{2}$ over the single arm solution for rearranging objects with non-overlapping starts and goals that are uniformly distributed in a unit square}
% 	\label{thm:karm}
% \end{theorem}

% \textit{Proof: }
% From the initial $2$-arm solution, we construct an {\em asynchronous} $2$-arm 
% solution that is collision-free. Assume that pickups and drop-offs can be achieved without collisions between the two arms, which can be achieved with properly designed end-effectors. The main overhead is then the potential collision between the two (disc) arms during transfer and move operations. Because there are $\frac{n}{2}$ objects for each arm to work 
% with, an arm may travel a path formed by $n + 1$ straight line 
% segments. Therefore, there are up to $(n + 1)^2$ intersections between the 
% two end-effector trajectories where potential collisions may happen. Nevertheless, since we can have at most four intersections for the transfers and transits associated with a pair of objects (one for each arm), there are at most $2n$ potential collisions to handle. For each intersection, let one 
% arm wait while the other circles around the first arm, which incurs a cost that is bounded by $2\pi \cdot r \cdot c_t$. 

% Adding up all the potential cost that a $2$-arm solution can incur, the following cumulative cost is obtained:
% % \begin{align}\label{eq:dual-cost-distance} 
% % C_{dual} = C_{single} + 2n(2\pi rc_t) \approx (c_{pd} + 0.52c_t + 4\pi rc_t)n.
% % \end{align}

% {\centerline
% {
% $C_{\rm dual} = C_{\rm single} + 2n(2\pi rc_t) \approx (c_{pd} + 0.52c_t + 4\pi rc_t)n\;.$
% }
% }

% \noindent For small $r$, $C_{\rm dual}$ is almost the same as $C_{\rm single}$, and
% $c_t$ is a distance (e.g., energy) cost. Upon considering the maximum of the two arc lengths or makespan (Eq.~\ref{eq:cost_function}),
% the $2$-arm cost becomes $C_{\rm dual}^t \approx (c_{pd} + 0.52c_t)\frac{n}{2} + 4n\pi rc_t$.
% % \vspace{-0.1in}
% % \begin{equation}\label{eq:dual-cost-time}
% % C_{dual}^t \approx (c_{pd} + 0.52c_t)\frac{n}{2} + 4n\pi rc_t
% % \vspace{-0.1in}
% % \end{equation}
% % {\centerline
% % {
% % $C_{dual}^t \approx (c_{pd} + 0.52c_t)\frac{n}{2} + 4n\pi rc_t$
% % }
% % }
% The cost ratio is
% %\vspace{-0.1in}
% \begin{align*}%\label{eq:makespan-ratio}
% \frac{C_{\rm dual}^t}{C_{\rm single}} &\approx 
% \frac{(c_{pd} + 0.52c_t)\frac{n}{2} + 4n\pi rc_t}{(c_{pd} + 0.52c_t)n}
% \\&= \frac{1}{2} + \frac{4\pi rc_t}{c_{pd} + 0.52c_t}\;\numberthis \label{eq:dualasync}
% % \vspace{-0.1in}
% \end{align*}



% \commentdel{
%  \begin{wrapfigure}{r}{1.4in}
% %  \vspace{-.45in}
% 	\centering
%     \includegraphics[width=1.4in]{figures/monte_carlo}
% %    \vspace{-.4in}
% 	\caption{Empirical cost ratio versus the estimate}
%   	\label{fig:bounds}
% %    \vspace{-.4in}
% \end{wrapfigure}
% }

% When $r$ is small or when $\frac{c_t}{c_{pd}}$ is small, the $2$-arm 
% solution is roughly half as costly as the single arm solution. On 
% the other hand, in this model a $2$-arm solution does not do better than $\frac{1}{2}$ of the single arm solution. $ \qed $

% %This argument can be extended to $k$-arms as well \cite{Shome2018WAFR}.



% %\vspace{-0.08in}
% \commentadd{
% %\begin{theorem}
% %For rearranging objects with non-overlapping starts and goals that are
% %uniformly distributed in a unit square,  a $2$-arm solution can have an 
% %asymptotic improvement of $\frac{1}{2}$ over the single arm solution. 
% %\label{thm:karm}
% %\end{theorem}
% }


% \tase{

% The following argument generalizes Theorem~\ref{thm:karm} to the case of $k$-arms.

% \begin{theorem}
% 	\textit{A $k$-arm solution can have an 
% 	asymptotic improvement of $\frac{1}{k}$ over the single arm solution, when rearranging objects with non-overlapping starts and goals that are uniformly distributed in a unit square.}
% \end{theorem}

% \textit{Proof:}	
% %\subsection{Expected k-arm cost bounds in a planar disk manipulator model}
% \commentadd{The arguments made in Theorem \ref{thm:karm} can be extended to $k$ disc arms.
% 	In the planar unit-square setting, with $k$ arms, there are $\frac{n}{k}$ objects 
% 	for each arm to work with. Consider the transfers and transits of a set of $k$ 
% 	objects, one for each arm. By \cite{ChiHanYu2018WAFR}, the arbitrary rearrangement of $k$ discs 
% 	can be achieved in a bounded region with a perimeter of $O(kr)$. 
% 	Clearly, the per robot additional (makespan or distance) cost is bounded by some 
% 	function $f(k, r)c_t$, which goes to zero as $r$ goes to zero. Adding up all the potential cost that can be incurred,  a $k$-arm solution has a cumulative cost of:
	
% 	{\centerline
% 		{
% 			$C_{\rm k{\text-}arm} = C_{\rm single} + nf(k,r)c_t \approx (c_{pd} + 0.52c_t + f(k,r)c_t)n\;.$
% 		}
% 	}
	
% 	\noindent For fixed $k$ and small $r$, $C_{\rm k{\text-}arm}$ is almost the same as $C_{\rm single}$.
% 	%, $c_t$ is a distance (e.g., energy) cost. 
% 	Upon considering the maximum of the two arc lengths or makespan,
% 	the $k$-arm cost becomes $C_{\rm k{\text-}arm}^t \approx (c_{pd} + 0.52c_t)\frac{n}{k} + nf(k,r)c_t$.
	
% 	The cost ratio is
% % 	\vspace{-0.1in}
% 	\begin{align*}\label{eq:kmakespan-ratio}
% 	\frac{C_{\rm k{\text-}arm}^t}{C_{\rm single}} &\approx 
% 	\frac{(c_{pd} + 0.52c_t)\frac{n}{k} + nf(k,r)c_t}{(c_{pd} + 0.52c_t)n}
% 	\\&= \frac{1}{k} + \frac{f(k,r)c_t}{c_{pd} + 0.52c_t}\;.\numberthis
% 	% \vspace{-0.1in}
% 	\end{align*}
	
% 	When $r$ is small or when $\frac{c_t}{c_{pd}}$ is small, the $k$-arm 
% 	solution is roughly $\frac{1}{k}$ as costly as the single arm solution. On 
% 	the other hand, in this model a $k$-arm solution does not do better than $\frac{1}{k}$ of 
% 	the single arm. \qed
	
% %	\begin{theorem}
% %		For rearranging objects with non-overlapping starts and goals that are
% %		uniformly distributed in a unit square,  a $k$-arm solution can have an 
% %		asymptotic improvement of $\frac{1}{k}$ over the single arm solution. 
% %	\end{theorem}
% }
% % \rahul{Need to verify this line of reasoning}

% }





This section studies the dual arm costs in the randomized unit tabletop setting, where $c_t$ is the cost measure per unit distance.  Assume for simplicity that each arm's volume is represented as a disc of some radius $r$. Eq.~\ref{eq:single-cost-simple} derives the cost estimate for a single arm solution as approximately $(c_{pd} + 0.52c_t)n$. Firstly the following arguments can be made for $k$-arms.


\tase{


\begin{theorem}
	\textit{A $k$-arm solution can have an 
	asymptotic improvement of $\frac{1}{k}$ over the single arm solution, when rearranging objects with non-overlapping starts and goals that are uniformly distributed in a unit square.}
\end{theorem}

\textit{Proof:}	
%\subsection{Expected k-arm cost bounds in a planar disk manipulator model}
\commentadd{%The arguments made in Theorem \ref{thm:karm} can be extended to $k$ disc arms.
	In the planar unit-square setting, with $k$ arms, there are $\frac{n}{k}$ objects 
	for each arm to work with. Consider the transfers and transits of a set of $k$ 
	objects, one for each arm. By \cite{ChiHanYu2018WAFR}, the arbitrary rearrangement of $k$ discs 
	can be achieved in a bounded region with a perimeter of $O(kr)$. 
	Clearly, the per robot additional (makespan or distance) cost is bounded by some 
	function $f(k, r)c_t$, which goes to zero as $r$ goes to zero. Adding up all the potential cost that can be incurred,  a $k$-arm solution has a cumulative cost of:
	
	{\centerline
		{
			$C_{\rm k{\text-}arm} = C_{\rm single} + nf(k,r)c_t \approx (c_{pd} + 0.52c_t + f(k,r)c_t)n\;.$
		}
	}
	
	\noindent For fixed $k$ and small $r$, $C_{\rm k{\text-}arm}$ is almost the same as $C_{\rm single}$.
	%, $c_t$ is a distance (e.g., energy) cost. 
	Upon considering the maximum of the two arc lengths or makespan,
	the $k$-arm cost becomes $C_{\rm k{\text-}arm}^t \approx (c_{pd} + 0.52c_t)\frac{n}{k} + nf(k,r)c_t$.
	
	The cost ratio is
% 	\vspace{-0.1in}
	\begin{align*}\label{eq:kmakespan-ratio}
	\frac{C_{\rm k{\text-}arm}^t}{C_{\rm single}} &\approx 
	\frac{(c_{pd} + 0.52c_t)\frac{n}{k} + nf(k,r)c_t}{(c_{pd} + 0.52c_t)n}
	\\&= \frac{1}{k} + \frac{f(k,r)c_t}{c_{pd} + 0.52c_t}\;.\numberthis
	% \vspace{-0.1in}
	\end{align*}
	
	When $r$ is small or when $\frac{c_t}{c_{pd}}$ is small, the $k$-arm 
	solution is roughly $\frac{1}{k}$ as costly as the single arm solution. On 
	the other hand, in this model a $k$-arm solution does not do better than $\frac{1}{k}$ of 
	the single arm. \qed
	

    }
}



For obtaining a $2$-arm solution, first partition the $n$ objects randomly into two sets of $\frac{n}{2}$ objects each. Then, obtain the two initial solutions similar to the single arm case.  
% It is expected (Eq.~\ref{eq:single-cost-simple}) that these two halves should add up to approximately $(c_{pd} + 0.52c_t)n$. 
% The reasoning follows Lemma \ref{lem:transferdomination}, which states that transfers dominate the cost of the solution.

\begin{corollary}
	\textit{A $2$-arm solution can have an asymptotic improvement of $\frac{1}{2}$ over the single arm solution for rearranging objects with non-overlapping starts and goals that are uniformly distributed in a unit square}
	\label{thm:karm}
\end{corollary}

\textit{Proof: }
From the initial $2$-arm solution, we construct an {\em asynchronous} $2$-arm 
solution that is collision-free. Assume that pickups and drop-offs can be achieved without collisions between the two arms, which can be achieved with properly designed end-effectors. The main overhead is then the potential collision between the two (disc) arms during transfer and move operations. Because there are $\frac{n}{2}$ objects for each arm to work 
with, an arm may travel a path formed by $n + 1$ straight line 
segments. 
\commentdel{Therefore, there are up to $(n + 1)^2$ intersections between the 
two end-effector trajectories where potential collisions may happen. Nevertheless,} 
Since we can have at most four intersections for the transfers and transits associated with a pair of objects (one for each arm), there are at most $2n$ potential collisions to handle. For each intersection, let one 
arm wait while the other circles around the first arm, which incurs a cost that is bounded by $2\pi \cdot r \cdot c_t$. 

Adding up all the potential cost that a $2$-arm solution can incur, the following cumulative cost is obtained:


{\centerline
{
$C_{\rm dual} = C_{\rm single} + 2n(2\pi rc_t) \approx (c_{pd} + 0.52c_t + 4\pi rc_t)n\;.$
}
}

\noindent For small $r$, $C_{\rm dual}$ is almost the same as $C_{\rm single}$, and
$c_t$ is a distance (e.g., energy) cost. Upon considering the maximum of the two arc lengths or makespan (Eq.~\ref{eq:cost_function}),
the $2$-arm cost becomes $C_{\rm dual}^t \approx (c_{pd} + 0.52c_t)\frac{n}{2} + 4n\pi rc_t$.

The cost ratio is
%\vspace{-0.1in}
\begin{align*}%\label{eq:makespan-ratio}
\frac{C_{\rm dual}^t}{C_{\rm single}} &\approx 
\frac{(c_{pd} + 0.52c_t)\frac{n}{2} + 4n\pi rc_t}{(c_{pd} + 0.52c_t)n}
\\&= \frac{1}{2} + \frac{4\pi rc_t}{c_{pd} + 0.52c_t}\;\numberthis \label{eq:dualasync}
% \vspace{-0.1in}
\end{align*}



\commentdel{
 \begin{wrapfigure}{r}{1.4in}
%  \vspace{-.45in}
	\centering
    \includegraphics[width=1.4in]{figures/monte_carlo}
%    \vspace{-.4in}
	\caption{Empirical cost ratio versus the estimate}
  	\label{fig:bounds}
%    \vspace{-.4in}
\end{wrapfigure}
}

When $r$ is small or when $\frac{c_t}{c_{pd}}$ is small, the $2$-arm 
solution is roughly half as costly as the single arm solution. On 
the other hand, in this model a $2$-arm solution does not do better than $\frac{1}{2}$ of the single arm solution. $ \qed $



















\begin{theorem}
	\textit{For rearranging objects with non-overlapping starts and goals that are 
	uniformly distributed in a unit square,  a randomized $2$-arm \textit{synchronized} solution can have an 
	asymptotic improvement of $\frac{1}{2}$ over the single arm solution if $\frac{c_t}{c_{pd}}$ is small, and an improvement of roughly $0.64$ when both $c_{pd}$ and $r$ are small. }
\label{thm:syncproof}
\end{theorem}


\textit{Proof:} 
\tase{
In preparation for the proof of Theorem \ref{thm:syncproof}, we develop the following lemma.
\begin{lemma}
	The expected measure of the maximum of lengths of two random lines on a unit square is $ 0.66 $.
\end{lemma}

\textit{Proof:} Prior work \cite{ghosh1951random} defines the \textit{probability distribution function (pdf)} of lengths($\ell$) of randomly sampled lines in a rectangle.
% of sides $a,b, \ \ a\geq b$.
% as 
%\begin{align*}
%p(l) &= (\frac{4l}{a^2b^2})\phi(l)\\
%\phi(l)&= \frac{1}{2}\pi a b - a l - b l + \frac{1}{2} l^2, \ \ l\in[0,b]\\
%\phi(l)&= ab \sininv(\frac{b}{l}) + a. \sqrt[]{(l^2-b^2)} - al - \frac{1}{2}b^2, \ \ l\in[b,a]\\
%\phi(l)&= ab\{\sininv(\frac{b}{l})-\cosinv(\frac{a}{l})\} + a\sqrt[]{(l^2-b^2)} + b\sqrt[]{(l^2-a^2)} \\
%&- \frac{1}{2}(l^2+a^2+b^2),\ \ l\in[a,\sqrt[]{(a^2+b^2)}]
%\end{align*}

%In the unit square model, $a=b=1$. Substituting the values, the \textit{pdf} becomes
Substituting the values for the dimensions of the rectangle in the unit square model, the \textit{pdf} can be simplified as follows.
\begin{align*}
p(\ell)&= 2\pi \ell - 8\pi \ell^2 + 2\ell^3, \ \ \ell\in[0,1]\\
p(\ell)&= 4\ell\sininv\!\!\left(\frac{1}{\ell}\right)\!- 4\ell\cosinv\!\!\left(\frac{1}{\ell}\right)\!+ 8\ell\sqrt{\ell^2-1} -2\ell^3 -4\ell,\\ &\ \ell\in[1,\sqrt[]{2}],
\end{align*}
where $\ell$ is the length measure, and $p(\ell)$ is the probability measure over different lengths.
Assuming two random sets of lines, representing transfers in a random split of objects between two arms, the expected value of the maximum of these pairwise lengths i.e., $\mathtt{E}(\max(\ell_1,\ell_2)), \ \ \ell_1,\ell_2\ i.i.d, \ \ \ell_1 \sim p, \ell_2 \sim p$, can be estimated using the \textit{pdf} obtained. 
\begin{align*}
\mathtt{E}(\max(\ell_1,\ell_2)) &= \int_0^{\sqrt[]{2}} \int_0^{\sqrt[]{2}} \max(\ell_1,\ell_2)p(\ell_1)p(\ell_2) d\ell_2 d\ell_1\\&\approx 0.663
\end{align*}
% \vspace{-0.1in}
% The result is calculated by taking into account the combination of different ranges of $p(\ell)$ and $\max(\ell_1,\ell_2)$.
\commentadd{
%	\begin{figure}[h]
%		\centering
%		%	\vspace{-0.3in}
%		\includegraphics[width=0.35\textwidth]{figures/monte_carlo}
%		\caption{Empirical cost ratio versus the estimate}
%		\label{fig:mcbounds}
%	\end{figure}
	Prior work~\cite{santalo2004integral} offered an estimate for the expected length of a transit path $C_{sg}$ in terms of the expected length of a line segment, $0.52$, in a randomized setting in a unit square. With the current estimate of $0.66$ for the maximum of two such randomly sampled line segments, it follows that, the expected makespan or maximum of distances cost will use this estimate.  
	
	\qed
	
%	Using this result, the synchronized cost ratio is stated in Equation~\ref{eq:synchronized-ratio} as
%	$$
%	\frac{C_{\rm dual}^{\rm sync}}{C_{\rm single}} \approx 
%	\frac{(c_{pd} + 0.66c_t)\frac{n}{2} + 4n\pi rc_t}{(c_{pd} + 0.52c_t)n}
%	$$
%	As a way to validate our asymptotic estimate, randomized trials were run with different number randomly sampled object transfer coordinates on an unit square.
%	% Fig \ref{fig:bounds} verifies empirically that the ratio of $\frac{C_{dual}^{sync}}{C_{single}}$ when $c_{pd}=0,r=0$, asymptotically converges to the calculated expected value of $0.636$. 
%	When $c_{pd}=0$ and $r=0$, the ratio of $\frac{C_{dual}^{sync}}{C_{single}}$ evaluates to $0.636$. Fig \ref{fig:mcbounds} verifies empirically that the ratio converges to the expected value as the number of transfers increases.
%	This indicates the asymptotic speedup of a synchronized dual arm solution for a makespan or maximum of distances cost metric.
}

}


Now all the tools necessary for the proof of Theorem \ref{thm:syncproof} are available.

The synchronization assumption changes the expected cost of the solution. The random partitioning of the $n$ objects into two sets of $\frac{n}{2}$ object with a random ordering of the objects yields $\frac{n}{2}$ pairs of objects transfers, which dominate the total cost for large $n$. The cost~(Eq.~\ref{eq:cost_function}) of $\frac{n}{2}$ synchronized transfers ($\coma_i$) includes $\frac{n}{2}c_{pd}$ and $C^{\rm sync}_{sg} \approx (\mathtt{E}(\max(\ell_1,\ell_2))c_t)\frac{n}{2}$, where $\mathtt{E}(\max(\ell_1,\ell_2))$ is the expected measure of the maximum of lengths $\ell_1$,$\ell_2$ of two randomly paired transfers. 
% This value is deduced in the next section.

%Using the \textit{pdf}\cite{ghosh1951random} of lengths of random lines in an unit square and integrating over the setup, results in the value of $\mathtt{E}(max(l_1,l_2))$ to be $0.66$ . 

% (Section \ref{sec:appendix}). 

It follows that $C_{\rm dual}^{\rm sync} \approx (c_{pd} + 0.66c_t)\frac{n}{2} + 4n\pi rc_t$.
% \vspace{-0.1in}
% \begin{equation}\label{eq:synchronized-cost}
% C_{dual}^{sync} \approx (c_{pd} + 0.66c_t)\frac{n}{2} + 4n\pi rc_t
% \vspace{-0.1in}
% \end{equation}
% {\centerline
% {
% $C_{dual}^{sync} \approx (c_{pd} + 0.66c_t)\frac{n}{2} + 4n\pi rc_t$
% }
% }
The synchronized cost ratio is
%\vspace{-0.1in}
\begin{align*}\label{eq:synchronized-ratio}
\frac{C_{\rm dual}^{\rm sync}}{C_{\rm single}} &\approx 
\frac{(c_{pd} + 0.66c_t)\frac{n}{2} + 4n\pi rc_t}{(c_{pd} + 0.52c_t)n}
\\&= \frac{1}{2} + \frac{ (0.07 + 4\pi r)c_t}{c_{pd} + 0.52c_t}\;.\numberthis
%\vspace{-0.1in}
\end{align*}

When $\frac{c_t}{c_{pd}}$ is small, even the synchronized $2$-arm solution provides an improvement of $\frac{1}{2}$. For the case when both $r$ and $c_{pd}$ are small, we observe that the ratio approaches $0.636$.\qed 
\commentdel{Fig \ref{fig:bounds} confirms empirically that in randomized trials on an unit square, the ratio of $\frac{C_{dual}^{sync}}{C_{single}}$ when $c_{pd}=0,r=0$, converges to the calculated estimate (red line).}

\tase{
	\begin{figure}[h]
		\centering
% 		\vspace{-0.1in}
		\includegraphics[width=0.35\textwidth]{figures/monte_carlo}
		\caption{ \tase{ Empirical cost ratio $\frac{C_{dual}^{sync}}{C_{single}}$ when $c_{pd}=0, r=0$ in the unit square model, versus the estimate (red line). As the number of object transfers increases the measured value converges to the estimate. } }
		\label{fig:mcbounds}
		\vspace{-0.1in}
	\end{figure}
	As a way to validate our asymptotic estimate, randomized trials were run with different number of randomly sampled object transfer coordinates on an unit square.
% Fig \ref{fig:bounds} verifies empirically that the ratio of $\frac{C_{dual}^{sync}}{C_{single}}$ when $c_{pd}=0,r=0$, asymptotically converges to the calculated expected value of $0.636$. 
When $c_{pd}=0$ and $r=0$, the ratio of $\frac{C_{dual}^{sync}}{C_{single}}$ evaluates to $0.636$. Fig \ref{fig:mcbounds} verifies empirically that the ratio converges to the expected value as the number of transfers increases.
This indicates the asymptotic speedup of a synchronized dual arm solution for a makespan or maximum of distances cost metric.
}

%\vspace{-0.08in}
%\begin{theorem}
%For rearranging objects with non-overlapping starts and goals that are 
%uniformly distributed in a unit square,  a randomized $2$-arm synchronized solution can have an 
%asymptotic improvement of $\frac{1}{2}$ over the single arm solution if $\frac{c_t}{c_{pd}}$ is small, and a improvement $\approx 0.64$ when both $c_{pd}$ and $r$ are small. 
%\end{theorem}


\commentadd{
\noindent \textbf{Note on bounds:} Even though the proposed simplified model may not be
immediately suitable for general configuration spaces, experiments indicate that the speedups exist in these spaces as well.
}
% \rahul{this paragraph is phrased contradictorily. It makes sense if the first sentence is for asynchronous cases.}
% We note that the same can be said for a {\em synchronous} $2$-arm solution if
% when $c_t/c_{pd}$ is small. 
% Under the synchronization assumption, $C_{dual}^t$
% is no longer necessarily expected to be $\frac{C_{single}}{2}$ for large $n$. 
% Experiments in the next section indicate that improvements over single arm solutions in practical settings.

%\textcolor{red}{NOTE: We can also generate plots illustrating how the ratio 
%change as $n$, $r$, $c_{pd}:c_t$ changes. We may also further generalize the 
%$r$ based collision term to a more general collision term.}


%\begin{figure}[h]
%%  \begin{center}
%\centering
%    \includegraphics[width=2in]{figures/bounds}
%%  \end{center}
%  \caption{Bounds}
%  	\label{fig:bounds}
%\end{figure}


